%_____________________________________________________________________________
%=============================================================================
% data.tex v10 (22-01-2017) dibuat oleh Lionov - T. Informatika FTIS UNPAR
%
% Perubahan pada versi 10 (22-01-2017)
%	- Penambahan overfullrule untuk memeriksa warning
%  	- perubahan mode buku menjadi 4: bimbingan, sidang(1), sidang akhir dan 
%     buku final
%	- perbaikan perintah pada beberapa bagian
%  	- perubahan pengisian tulisan "daftar isi" yang error
%  	- penghilangan lipsum dari file ini
%_____________________________________________________________________________
%=============================================================================

%=============================================================================
% 								PETUNJUK
%=============================================================================
% Ini adalah file data (data.tex)
% Masukkan ke dalam file ini, data-data yang diperlukan oleh template ini
% Cara memasukkan data dijelaskan di setiap bagian
% Data yang WAJIB dan HARUS diisi dengan baik dan benar adalah SELURUHNYA !!
% Hilangkan tanda << dan >> jika anda menemukannya
%=============================================================================

%_____________________________________________________________________________
%=============================================================================
% 								BAGIAN 0
%=============================================================================
% Entri untuk memperbaiki posisi "DAFTAR ISI" jika tidak berada di bagian 
% tengah halaman. Sayangnya setiap sistem menghasilkan posisi yang berbeda.
% Isilah dengan 0 atau 1 (e.g. \daftarIsiError{1}). 
% Pemilihan 0 atau 1 silahkan disesuaikan dengan hasil PDF yang dihasilkan.
%=============================================================================
\daftarIsiError{0}   
%\daftarIsiError{1}   
%=============================================================================

%_____________________________________________________________________________
%=============================================================================
% 								BAGIAN I
%=============================================================================
% Tambahkan package2 lain yang anda butuhkan di sini
%=============================================================================
\usepackage{booktabs} 
\usepackage{longtable}
\usepackage{amssymb}
\usepackage{todo}
\usepackage{verbatim} 		%multiline comment
\usepackage{pgfplots}
\usepackage{array}
\usepackage[boxruled]{algorithm2e}
\usepackage[inline]{enumitem}
\usepackage{float}
\usepackage{float}
\usepackage{mathptmx}
%\overfullrule=3mm % memperlihatkan overfull 
%=============================================================================

%_____________________________________________________________________________
%=============================================================================
% 								BAGIAN II
%=============================================================================
% Mode dokumen: menetukan halaman depan dari dokumen, apakah harus mengandung 
% prakata/pernyataan/abstrak dll (termasuk daftar gambar/tabel/isi) ?
% - final 		: hanya untuk buku skripsi, dicetak lengkap: judul ina/eng, 
%   			  pengesahan, pernyataan, abstrak ina/eng, untuk, kata 
%				  pengantar, daftar isi (daftar tabel dan gambar tetap 
%				  opsional dan dapat diatur), seluruh bab dan lampiran.
%				  Otomatis tidak ada nomor baris dan singlespacing
% - sidangakhir	: buku sidang akhir = buku final - (pengesahan + pernyataan +
%   			  untuk + kata pengantar)
%				  Otomatis ada nomor baris dan onehalfspacing 
% - sidang 		: untuk sidang 1, buku sidang = buku sidang akhir - (judul 
%				  eng + abstrak ina/eng)
%				  Otomatis ada nomor baris dan onehalfspacing
% - bimbingan	: untuk keperluan bimbingan, hanya terdapat bab dan lampiran
%   			  saja, bab dan lampiran yang hendak dicetak dapat ditentukan 
%				  sendiri (nomor baris dan spacing dapat diatur sendiri)
% Mode default adalah 'template' yang menghasilkan isian berwarna merah, 
% aktifkan salah satu mode di bawah ini :
%=============================================================================
%\mode{bimbingan} 		% untuk keperluan bimbingan
%\mode{sidang} 			% untuk sidang 1
\mode{sidangakhir} 	% untuk sidang 2 / sidang pada Skripsi 2(IF)
%\mode{final} 			% untuk mencetak buku skripsi 
%=============================================================================

%_____________________________________________________________________________
%=============================================================================
% 								BAGIAN III
%=============================================================================
% Line numbering: penomoran setiap baris, nomor baris otomatis di-reset ke 1
% setiap berganti halaman.
% Sudah dikonfigurasi otomatis untuk mode final (tidak ada), mode sidang (ada)
% dan mode sidangakhir (ada).
% Untuk mode bimbingan, defaultnya ada (\linenumber{yes}), jika ingin 
% dihilangkan, isi dengan "no" (i.e.: \linenumber{no})
% Catatan:
% - jika nomor baris tidak kembali ke 1 di halaman berikutnya, compile kembali
%   dokumen latex anda
% - bagian ini hanya bisa diatur di mode bimbingan
%=============================================================================
%\linenumber{no} 
\linenumber{yes}
%=============================================================================

%_____________________________________________________________________________
%=============================================================================
% 								BAGIAN IV
%=============================================================================
% Linespacing: jarak antara baris 
% - single	: otomatis jika ingin mencetak buku skripsi, opsi yang 
%			     disediakan untuk bimbingan, jika pembimbing tidak keberatan 
%			     (untuk menghemat kertas)
% - onehalf	: otomatis jika ingin mencetak dokumen untuk sidang
% - double 	: jarak yang lebih lebar lagi, jika pembimbing berniat memberi 
%             catatan yg banyak di antara baris (dianjurkan untuk bimbingan)
% Catatan: bagian ini hanya bisa diatur di mode bimbingan
%=============================================================================
\linespacing{single}
%\linespacing{onehalf}
%\linespacing{double}
%=============================================================================

%_____________________________________________________________________________
%=============================================================================
% 								BAGIAN V
%=============================================================================
% Tidak semua skripsi memuat gambar dan/atau tabel. Untuk skripsi yang tidak 
% memiliki gambar dan/atau tabel, maka tidak diperlukan Daftar Gambar dan/atau 
% Daftar Tabel. Sayangnya hal tsb sulit dilakukan secara manual karena 
% membutuhkan kedisiplinan pengguna template.  
% Jika tidak ingin menampilkan Daftar Gambar dan/atau Daftar Tabel, karena 
% tidak ada gambar atau tabel atau karena dokumen dicetak hanya untuk 
% bimbingan, isi dengan "no" (e.g. \gambar{no})
%=============================================================================
\gambar{yes}
%\gambar{no}
\tabel{yes}
%\tabel{no}  
%=============================================================================

%_____________________________________________________________________________
%=============================================================================
% 								BAGIAN VI
%=============================================================================
% Pada mode final, sidang da sidangkahir, seluruh bab yang ada di folder "Bab"
% dengan nama file bab1.tex, bab2.tex s.d. bab9.tex akan dicetak terurut, 
% apapun isi dari perintah \bab.
% Pada mode bimbingan, jika ingin:
% - mencetak seluruh bab, isi dengan 'all' (i.e. \bab{all})
% - mencetak beberapa bab saja, isi dengan angka, pisahkan dengan ',' 
%   dan bab akan dicetak terurut sesuai urutan penulisan (e.g. \bab{1,3,2}). 
% Catatan: Jika ingin menambahkan bab ke-3 s.d. ke-9, tambahkan file 
% bab3.tex, bab4.tex, dst di folder "Bab". Untuk bab ke-10 dan 
% seterusnya, harus dilakukan secara manual dengan mengubah file skripsi.tex
% Catatan: bagian ini hanya bisa diatur di mode bimbingan
%=============================================================================
\bab{4,5}
%=============================================================================

%_____________________________________________________________________________
%=============================================================================
% 								BAGIAN VII
%=============================================================================
% Pada mode final, sidang dan sidangkhir, seluruh lampiran yang ada di folder 
% "Lampiran" dengan nama file lampA.tex, lampB.tex s.d. lampJ.tex akan dicetak 
% terurut, apapun isi dari perintah \lampiran.
% Pada mode bimbingan, jika ingin:
% - mencetak seluruh lampiran, isi dengan 'all' (i.e. \lampiran{all})
% - mencetak beberapa lampiran saja, isi dengan huruf, pisahkan dengan ',' 
%   dan lampiran akan dicetak terurut sesuai urutan (e.g. \lampiran{A,E,C}). 
% - tidak mencetak lampiran apapun, isi dengan "none" (i.e. \lampiran{none})
% Catatan: Jika ingin menambahkan lampiran ke-C s.d. ke-I, tambahkan file 
% lampC.tex, lampD.tex, dst di folder Lampiran. Untuk lampiran ke-J dan 
% seterusnya, harus dilakukan secara manual dengan mengubah file skripsi.tex
% Catatan: bagian ini hanya bisa diatur di mode bimbingan
%=============================================================================
\lampiran{4,5}
%=============================================================================

%_____________________________________________________________________________
%=============================================================================
% 								BAGIAN VIII
%=============================================================================
% Data diri dan skripsi/tugas akhir
% - namanpm		: Nama dan NPM anda, penggunaan huruf besar untuk nama harus 
%				  benar dan gunakan 10 digit npm UNPAR, PASTIKAN BAHWA 
%				  BENAR !!! (e.g. \namanpm{Jane Doe}{1992710001}
% - judul 		: Dalam bahasa Indonesia, perhatikan penggunaan huruf besar, 
%				  judul tidak menggunakan huruf besar seluruhnya !!! 
% - tanggal 	: isi dengan {tangga}{bulan}{tahun} dalam angka numerik, 
%				  jangan menuliskan kata (e.g. AGUSTUS) dalam isian bulan.
%			  	  Tanggal ini adalah tanggal dimana anda akan melaksanakan 
%				  sidang ujian akhir skripsi/tugas akhir
% - pembimbing	: pembimbing anda, lihat daftar dosen di file dosen.tex
%				  jika pembimbing hanya 1, kosongkan parameter kedua 
%				  (e.g. \pembimbing{\JND}{} ), \JND adalah kode dosen
% - penguji 	: para penguji anda, lihat daftar dosen di file dosen.tex
%				  (e.g. \penguji{\JHD}{\JCD} )
% !!Lihat singkatan pembimbing dan penguji anda di file dosen.tex!!
% Petunjuk: hilangkan tanda << & >>, dan isi sesuai dengan data anda
%=============================================================================
\namanpm{Matthew Ariel}{2015730010}
\tanggal{25}{11}{2019}
\pembimbing{\VSM}{}    
\penguji{<<penguji 1>>}{<<penguji 2>>} 
%=============================================================================

%_____________________________________________________________________________
%=============================================================================
% 								BAGIAN IX
%=============================================================================
% Judul dan title : judul bhs indonesia dan inggris
% - judulINA: judul dalam bahasa indonesia
% - judulENG: title in english
% Petunjuk: 
% - hilangkan tanda << & >>, dan isi sesuai dengan data anda
% - langsung mulai setelah '{' awal, jangan mulai menulis di baris bawahnya
% - gunakan \texorpdfstring{\\}{} untuk pindah ke baris baru
% - judul TIDAK ditulis dengan menggunakan huruf besar seluruhnya !!
%=============================================================================
\judulINA{Reduksi Big Data dengan Algoritma Agglomerative Clustering untuk Spark}
\judulENG{Big Data Reduction with Agglomerative Clustering Algorithm for Spark}
%_____________________________________________________________________________
%=============================================================================
% 								BAGIAN X
%=============================================================================
% Abstrak dan abstract : abstrak bhs indonesia dan inggris
% - abstrakINA: abstrak bahasa indonesia
% - abstrakENG: abstract in english 
% Petunjuk: 
% - hilangkan tanda << & >>, dan isi sesuai dengan data anda
% - langsung mulai setelah '{' awal, jangan mulai menulis di baris bawahnya
%=============================================================================
\abstrakINA{
{\it Big data} adalah istilah yang menggambarkan kumpulan data dalam jumlah yang sangat besar, baik data yang terstruktur maupun data yang tidak terstruktur. Kumpulan data tersebut menyimpan informasi yang bisa dianalisis dan diproses untuk memberikan wawasan kepada organisasi atau perusahaan. {\it Big data} dapat mencapai \textit{petabyte} dan menghabiskan banyak tempat penyimpanan.\\

\textit{Big data} perlu direduksi untuk menghemat tempat penyimpanan. Algoritma \textit{Hierarchical Agglomerative Clustering} dapat digunakan untuk mereduksi data. Dengan bantuan sistem terdistribusi seperti Hadoop, proses reduksi data dapat dilakukan secara paralel dan lebih cepat. Sayangnya, teknologi Hadoop masih dapat dikatakan 'terlalu lambat' dalam melakukan proses reduksi data karena hasil sementara dari setiap tahap akan disimpan di \textit{disk} sampai dibutuhkan kembali di tahap selanjutnya.\\

Untuk mempercepat proses reduksi data, Hadoop dapat digantikan dengan Spark. Spark adalah sistem terdistribusi, mirip seperti Hadoop. Tetapi, yang membedakan antara Hadoop dengan Spark adalah pada cara penyimpanan sementara saat melakukan proses reduksi data. Hadoop menggunakan \textit{disk} sebagai tempat penyimpanan sementaranya, sedangkan Spark menggunakan memori sebagai tempat penyimpanan sementaranya. Pembacaan dan penulisan akan lebih cepat saat menggunakan memori dibandingkan dengan menggunakan \textit{disk}, sehingga Spark akan lebih cepat dibandingkan dengan Hadoop.\\

Perangkat lunak dibuat untuk mengimplementasikan algoritma \textit{Hierarchical Agglomerative Clustering} dalam Spark. Pengujian juga dilakukan dengan membandingkan waktu eksekusi algoritma \textit{Hierarchical Agglomerative Clustering} saat diimplementasikan pada Hadoop dan saat diimplementasikan pada Spark. Waktu eksekusi dicatat untuk ukuran data 1GB, 2GB, 3GB, 5GB, 10GB, 15GB, dan 20GB.\\

Berdasarkan hasil pengujian, Spark memiliki waktu eksekusi yang lebih cepat dibandingkan dengan Hadoop pada jumlah partisi yang besar. Waktu eksekusi Spark menurun ketika jumlah partisi ditingkatkan, sedangkan waktu eksekusi Hadoop menurun ketika jumlah partisi ditingkatkan. Waktu eksekusi terbaik Spark masih lebih cepat dibandung waktu eksekusi terbaik Hadoop.\\

}

\abstrakENG{
Big data is a term that describes the large volume of data, both structured and unstructured. The data set stores informationcan be analyzed and processed to provide insight to organization or company. Big data can reach up to petabytes and takes a lot of storage spaces.\\

Big data need to be reduce to save storage space. The Hierarchical Agglomerative Clustering algorithm can be used to reduce data. With the help of distributed systems such as Hadoop, reduction process can be done in parallel with less execution time. Unfortunately, Hadoop can still be said to be 'too slow' in the process of data reduction because temporary results from each stage will be stored on the disk until it is needed again at a later stage.\\

To speed up the data reduction process, Hadoop can be replaced with Spark. Spark is a distributed system, similar to Hadoop. However, what distinguishes Hadoop from Spark is the way Spark temporarily store data. Hadoop uses disk as its temporary storage, while Spark uses memory as its temporary storage. Readi and write process will be faster when using memory than using disks, Spark will be faster than Hadoop.\\

The Hierarchical Agglomerative Clustering algorithm is implemented in the software. Experiment were done by comparing the execution time of the  Hierarchical Agglomerative Clustering algorithm when implemented on Hadoop and Spark. The execution time is recorded for 1GB, 2GB, 3GB, 5GB, 10GB, 15GB, dan 20GB of data.\\

Based on the experiment, Spark has a faster execution time compared to Hadoop on a large number of partitions. Spark execution time decreases when the number of partitions is increased, whereas Hadoop execution time decreases when the number of partitions is increased. Spark best executiron time is still much better than Hadoop best execution time.\\
} 
%=============================================================================

%_____________________________________________________________________________
%=============================================================================
% 								BAGIAN XI
%=============================================================================
% Kata-kata kunci dan keywords : diletakkan di bawah abstrak (ina dan eng)
% - kunciINA: kata-kata kunci dalam bahasa indonesia
% - kunciENG: keywords in english
% Petunjuk: hilangkan tanda << & >>, dan isi sesuai dengan data anda.
%=============================================================================
\kunciINA{
\textit{Big Data}, Reduksi Data, \textit{Hierarchical Agglomerative Clustering}, 
Spark, Hadoop}
\kunciENG{Big Data, Data Reduction, Hierarchical Agglomerative Clustering, Spark, Hadoop}
%=============================================================================

%_____________________________________________________________________________
%=============================================================================
% 								BAGIAN XII
%=============================================================================
% Persembahan : kepada siapa anda mempersembahkan skripsi ini ...
% Petunjuk: hilangkan tanda << & >>, dan isi sesuai dengan data anda.
%=============================================================================
\untuk{Dipersembahkan untuk keluarga, teman, Ibu
Veronica, dan diri sendiri}
%=============================================================================

%_____________________________________________________________________________
%=============================================================================
% 								BAGIAN XIII
%=============================================================================
% Kata Pengantar: tempat anda menuliskan kata pengantar dan ucapan terima 
% kasih kepada yang telah membantu anda bla bla bla ....  
% Petunjuk: hilangkan tanda << & >>, dan isi sesuai dengan data anda.
%=============================================================================
\prakata{
Puji dan syukur kehadirat Tuhan Yang Maha Esa atas berkat rahmat serta kasih-Nya sehingga penulis dapat menyelesaikan skripsi ini yang mengambil judul “Reduksi Big Data dengan Algoritma Agglomerative Clustering untuk Spark”. Penulisan skripsi ini diajukan untuk memenuhi salah satu syarat untuk memperoleh gelar Sarjana
pada Program Studi Teknik Informatika Universitas Katolik Parahyangan. Pada penyusunan dan
penulisan skripsi ini, penulis menyadari bahwa penyusunan dan penulisan skripsi ini juga tidak
terlepas dari bantuan berbagai pihak, baik langsung maupun tidak langsung. Secara khusus, penulis
ingin berterima kasih kepada:

\begin{enumerate}
\item Keluarga yang selalu memberi dukungan dan mengurus penulis selama mengerjakan skripsi.

\item Ibu \VSM selaku dosen pembimbing yang telah membimbing penulis dan memberikan dukungan dan bantuan kepada penulis dalam proses penyusunan
skripsi ini.

\item Teman dan kerabat lainya yang tidak bisa saya sebutkan satu-persatu.

\end{enumerate}

Penulis menyadari bahwa skripsi ini masih jauh dari kata sempurna. Oleh karena itu, penu-
lis memohon maaf jika terdapat kesalahan. Penulis juga mengharapkan kritik dan saran yang
membangun untuk menyempurnakan skripsi ini. Semoga skripsi ini dapat memberi informasi yang
bermanfaat dan menjadi inspirasi untuk penelitian-penelitian berikutnya.
} 
%=============================================================================

%_____________________________________________________________________________
%=============================================================================
% 								BAGIAN XIV
%=============================================================================
% Tambahkan hyphen (pemenggalan kata) yang anda butuhkan di sini 
%=============================================================================
\hyphenation{ma-te-ma-ti-ka}
\hyphenation{fi-si-ka}
\hyphenation{tek-nik}
\hyphenation{in-for-ma-ti-ka}
%=============================================================================

%_____________________________________________________________________________
%=============================================================================
% 								BAGIAN XV
%=============================================================================
% Tambahkan perintah yang anda buat sendiri di sini 
%=============================================================================
\renewcommand{\vtemplateauthor}{lionov}
\pgfplotsset{compat=newest}
%=============================================================================
